\section{Introduction}
\label{section:introduction}

Today many organizations are having multiple branch offices that are 
geographically separated. Connecting these offices with secure 
communication channels is a must. The number of offices can be a large
number and, thus, scalability is another crucial requirement. 

We consider a multipoint L3-VPN in this work. The routers which are 
part of this VPN are arranged in such a manner (hub-and-spoke arrangement) 
so that scalability requirement is fulfilled. To secure the network we 
use Host Identity Protocol~\cite{gurtov:hip}. We consider that routers 
perform so called base exchange to negotiate authentication key
on hop-by-hop bases (a separate base exchange is performed between connected pairs of nodes). 
Although, encryption is prerogative of the customers,
packet authentication is performed in a hop-by-hop manner by the routers on 
the path. To understand the performance  of the proposed architecture we
compare two setups: one with hop-by-hop packet authentication and the 
second without authentication header attached to the packets at every 
hop in L3-VPN network. Finally, we perform brief analysis of various approaches
for building VPLS networks. 

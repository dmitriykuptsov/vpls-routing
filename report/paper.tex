\documentclass[conference,10pt,letter]{IEEEtran}

\usepackage{url}
\usepackage{amssymb,amsthm}
\usepackage{graphicx,color}

%\usepackage{balance}

\usepackage{cite}
\usepackage{amsmath}
\usepackage{amssymb}

\usepackage{color, colortbl}
\usepackage{times}
\usepackage{caption}
\usepackage{rotating}
\usepackage{subcaption}

\newtheorem{theorem}{Theorem}
\newtheorem{example}{Example}
\newtheorem{definition}{Definition}
\newtheorem{lemma}{Lemma}

\newcommand{\XXXnote}[1]{{\bf\color{red} XXX: #1}}
\newcommand{\YYYnote}[1]{{\bf\color{red} YYY: #1}}
\newcommand*{\etal}{{\it et al.}}

\newcommand{\eat}[1]{}
\newcommand{\bi}{\begin{itemize}}
\newcommand{\ei}{\end{itemize}}
\newcommand{\im}{\item}
\newcommand{\eg}{{\it e.g.}\xspace}
\newcommand{\ie}{{\it i.e.}\xspace}
\newcommand{\etc}{{\it etc.}\xspace}

\def\P{\mathop{\mathsf{P}}}
\def\E{\mathop{\mathsf{E}}}

\begin{document}
\sloppy
\title{Building scalable and secure multipoint {{L3-VPN}}: Mininet prototype }
\maketitle
\begin{abstract}
In this short document we describe scalable and secure multipoint L3-VPN architecture.
We implement the prototype in Mininet framework and evaluate end-to-end 
performance of the hosts which are part of the L3-VPN. We consider two setups:
with hop-by-hop authentication and without. Furthermore, we consider that the
authentication keys are distributed with Host Identity Protocol, and that the 
routers authenticate themselves using public keys. We conclude the paper with 
comparision of different approaches for building the VPLS networks.
\end{abstract}
\input intro.tex
\input architecture.tex
\input experimental.tex
%\input conclusions.tex
%\balance
\bibliographystyle{abbrv}
\bibliography{mybib}

\end{document}

